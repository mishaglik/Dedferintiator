\chapter{Матчасть}
\epigraph{{А прежде всего матчасть.}}{\textit{Народная мудрость}}
\section{О числах}
\subsection{Целые числа}
Все вы знаете целые числа. А вот вам вопрос $ -1/12 $ целое ли число?
Теперь всё сложно ведь:
$$
\sum_{n=1}^{\infty} n = -\frac{1}{12}
$$
Слева целое число, как сумма целых. А вот справа нет. Па-ла-докс!
Вообще нам за глаза хватает целых чисел\footnote{А есть быть точным то диапазона 
	$ \left[-2^{31};2^{31}\right) $}. Но вот открою вам страшную тайну. Нет никакой необходимости вводить другие.
\subsection{Вещественные числа}
Их нет.

 Вообще.
 
 \;\;\;\; Совсем.
