\documentclass{report}
\title{\textbf{\Huge Матан?}\\ {\LARGE Символьные методы дифференциального исчисления над множеством функций многих переменных.}\\ {\small 1.5 Производные по цене 1}}
\usepackage[utf8]{inputenc}
\usepackage[T2A]{fontenc}
\usepackage[russian]{babel}
\usepackage{amssymb}
\usepackage[auto]{chappg}\usepackage{epigraph}\usepackage[normalem]{ulem}
\usepackage{amsmath}
\usepackage{geometry}
\usepackage{enumitem}
%\newgeometry{vmargin={15mm}, hmargin={12mm,17mm}}
\author{Глинский Михаил Б05-133 }
\date{Ноябрь. Осень. Дождь. Снег. Мокрый снег. 2021}
\begin{document}
\maketitle
\vspace*{\fill}
\begin{center}
\section*{Предисловие}
\end{center}
\vspace*{\fill}
Дорогой читатель! Читая эту ***\footnote{Расшифровка эвфеминистических выражений предоставляется читателю в качестве упражнения. - Авт.} ты и представить себе не можешь сколько \sout{боли и страданий} удовольствия ты получишь в результате прочтения!
Я, великий автор книги, ставил целью максимально просто и понятно донести до читателя основы излагаемого материала.
Но как бы то ни было, нельзя вот так взять и понять матан. В математике царит баланс. И для обретения знаний нужно потерять что-то равноценное. \textit{Душу, например}.

Прочтя эту \%!@* ты осознаешь, почему не стоит трать время впустую... Итак, вперёд, читатель!!!

\begin{flushright}
\textit{C уважением, Автор}
\end{flushright}
\vspace*{\fill}
\tableofcontents