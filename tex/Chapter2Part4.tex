\section{Упражнения}
В качестве упражнения читателю предлагается найти несколько производных самоcтоятельно.

\begin{enumerate}
	\item $ f(x_1,x_2,x_3,\dots) = x_1^{x_2^{x_3^{\dots}}}$
	\item $ f(x) = \frac{\sin{1/x} * e^{x \ctg{x^2}}}{1 - x^{|x|}} $
	\item $ f(x) = \frac{x^y + y^x}{z^z} $
 	\item $ f(x) = \left({
 	\frac{ \sin^2{(3x-1)} + \sqrt{x^4 - \ln x}}
 	{5^{x^7} - \ch(\arccos(x))}
 }\right)^{x\tg x}$\cite{Semestr}
\end{enumerate}
\chapter{Г-н Тейлор}.
\section{Определение}
\epigraph{"Да я как разложу тебя по ортонормированному базису в линейном пространстве"}{{\textit Average linear algebra enjoyer}}

Разложением по Тейлору зовется выражение следующего вида:

$$
	f(x) = \sum_{k=1}^{n} \frac{f^{(k)}(x_0)  (x-x_0)^k}{k!} + o(x^n)
$$

\section{Пример}
Для примера разложим функцию $ x * \sin(x) $ до $ o(x^5)$ при $ x \rightarrow 0 $